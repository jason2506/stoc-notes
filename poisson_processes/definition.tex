\section{Definition of Poisson Process}

A continuous-time stochastic process
\[ \mathbb{N} = \{ N(t) \mid t \ge 0 \} \]
is a \defterm{counting process} if $ N(t) $ denotes the total number of ``events'' that occur by time $ t $.
\begin{comment}
$ N(t) $: non-negative integer, increases as $ t $ increases.
\end{comment}

\begin{definition}[\defterm{independent increments}]
The number of events that occur in disjoint time intervals are independent.
\end{definition}

\begin{definition}[\defterm{stationary increments}]
The distribution of the number of events that occur in any interval of time depends only the length of the time interval.
\end{definition}

\begin{definition}[\defterm{Poisson process}]
A \defterm*{poisson process} with rate $ \lambda $ is a counting process $ \mathbb{N} $ with $ N(0) = 0 $ possessing (a) independent increments and (b) stationary increments such that the number of events in any interval of length $ t $ is a poisson distribution with mean $ \lambda t $. That is,
\[ P(N(t + s) - N(s) = n) = \frac{e^{-\lambda t} \cdot (\lambda t)^{n}}{n!}. \]
\end{definition}

\begin{theorem}[操作型定義] \label{thm:op_def}
$ \mathbb{N} $ is a possion process iff $ \mathbb{N} $ possesses independent increments and stationary increments and the following two conditions for any $ t \ge 0 $ holds:
\begin{enumerate}
  \item $ P(N(t) = 1) = \lambda t + o(t) $,
  \item $ P(N(t) \ge 2) = o(t) $,
\end{enumerate}
where a function $ f(t) $ is $ o(t) $ if $ \lim_{t \to 0} f(t) / t = 0 $.

\begin{proof}[Proof (only if)]
\begin{eqnarray*}
P(N(t) = 1)
  & = & e^{-\lambda t} \cdot (\lambda t) \\
  & = & \lambda t \cdot \left( 1 - \frac{\lambda t}{1!} + \frac{(\lambda t)^{2}}{2!} - \frac{(\lambda t)^{3}}{3!} + \cdots \right) \\
  & = & \lambda t + o(t), \\
P(N(t) = 2)
  & = & \frac{e^{-\lambda t} \cdot (\lambda t)^{2}}{2!} \\
  & = & (\lambda t + o(t)) \cdot \frac{\lambda t}{2} \\
  & = & o(t).
\end{eqnarray*}
\end{proof}

\begin{proof}[Proof (if)]
If $ X $ is Poisson distribution with parameter $ \lambda t $,
\begin{alignat*}{3}
\psi(u) & \quad=\quad & E[e^{ux}] & \quad=\quad & e^{\lambda t (e^{u} - 1)} & \quad\text{(moment generating function)} \\
\phi(u) & \quad=\quad & E[e^{-ux}] & \quad=\quad & e^{\lambda t (e^{-u} - 1)} & \quad\text{(Laplace transform)}
\end{alignat*}

Define $ \phi_{u}(t) = E[e^{-uN(t)}] $ is the Laplace transform of $ N(t) $ that satisfies conditions 1 \& 2.
\begin{eqnarray*}
\phi_{u}(s + t)
  & = & E[e^{-uN(s + t)}] \\
  & = & E[e^{-uN(s)} \cdot e^{-u(N(s + t) - N(s))}] \\
  & = & E[e^{-uN(s)}] \cdot E[e^{-u(N(s + t) - N(s))}] \\
  & = & E[e^{-uN(s)}] \cdot E[e^{-uN(t)}] \\
  & = & \phi_{u}(s) \cdot \phi_{u}(t).
\end{eqnarray*}

By conditions 1 \& 2, we have
\[ P(N(t) = 0) = 1 - \lambda t + o(t). \]

Therefore,
\begin{eqnarray*}
\phi_{u}(t)
  & = & E[e^{-uN(t)}] \\
  & = & \sum_{n = 0}^{\infty} e^{-un} \cdot P(N(t) = n) \\
  & = & (1 - \lambda t + o(t)) + e^{-u} \cdot (\lambda t + o(t)) + e^{-2u} \cdot o(t) + e^{-3u} \cdot o(t) + \cdots \\
  & = & 1 + (e^{-u} - 1) \cdot \lambda t + o(t).
\end{eqnarray*}

Thus,
\begin{eqnarray*}
\phi_{u}(s + t)
  & = & \phi_{u}(s) \cdot \phi_{u}(t) \\
  & = & \phi_{u}(s) \cdot [1 + (e^{-u} - 1) \cdot \lambda t + o(t)].
\end{eqnarray*}

That is,
\begin{eqnarray*}
\phi_{u}'(s)
  & = & \lim_{t \to 0} \frac{\phi_{u}(s + t) - \phi_{u}(s)}{t} \\
  & = & \phi_{u}(s) \cdot (e^{-u} - 1) \cdot \lambda + \lim_{t \to 0} \frac{o(t)}{t} \\
  & = & \phi_{u}(s) \cdot (e^{-u} - 1) \cdot \lambda, \\
\frac{\phi_{u}'(s)}{\phi_{u}(s)}
  & = & (e^{-u} - 1) \cdot \lambda.
\end{eqnarray*}
and hence
\[ \ln \phi_{u}(s) = \int [(e^{-u} - 1) \cdot \lambda] \mathrm{d} s = (e^{-u} - 1) \cdot \lambda s + C. \]
In addition, since $ \phi_{u}(0) = 1 $, we get $ C = 0 $ and hence $ \phi_{u}(s) = e^{(e^{-u} - 1) \cdot \lambda s} $
\end{proof}
\end{theorem}

\begin{exercise}
利用 \autoref{thm:op_def} 證明
\begin{enumerate}
  \item If $ \mathbb{N}_{1} $ is a Poisson process with rate $ \lambda_{1} $, and $ \mathbb{N}_{2} $ is a Poisson process with rate $ \lambda_{2} $, then $ \mathbb{N}_{1} \cup \mathbb{N}_{2} $ is a Poisson process with rate $ \lambda_{1} + \lambda_{2} $.
  \item If $ X $ is a Poisson distribution with parameter $ \lambda_{1} $, and $ Y $ is a Poisson distribution with parameter $ \lambda_{2} $ ($ X $ and $ Y $ are independent), then $ X+Y $ is a Poisson distribution with parameter $ \lambda_{1} + \lambda_{2} $.
\end{enumerate}
\end{exercise}
