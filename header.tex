% ===============================================
%   Package
% ===============================================

\usepackage[dvipsnames]{xcolor}
\usepackage[colorlinks,linkcolor=NavyBlue]{hyperref}

\usepackage{CJK}
\usepackage{amsthm, amsmath, amssymb}
\usepackage{etoolbox}
\usepackage{fancyhdr}
\usepackage{import}
\usepackage{makeidx}
\usepackage{multicol}
\usepackage{thmtools}
\usepackage{tikz}
\usepackage{xparse}

% ===============================================
%   Import
% ===============================================
\newcommand{\ImportChapter}[1]{\import{#1/}{#1/main}}

% ===============================================
%   Layout
% ===============================================

\pagestyle{fancy}
\topmargin=-0.45in
\textheight=9.5in
\advance\textwidth by 1.5in
\advance\oddsidemargin by -0.75in
\advance\evensidemargin by -0.75in

\linespread{1.2}
\setlength{\headwidth}{\textwidth}

% ===============================================
%   tikz
% ===============================================

\usetikzlibrary{arrows,positioning,automata}
\tikzset{->,>=latex,shorten >=1pt,auto,node distance=5em}
\tikzstyle{every state}=[draw=black,text=black,minimum size=3em]

% ===============================================
%   Theorem
% ===============================================

\theoremstyle{plain}
\newtheorem{theorem}{\color{BrickRed} Theorem}[chapter]
\newtheorem{lemma}{\color{BrickRed} Lemma}[chapter]
\newtheorem{observation}{\color{BrickRed} Observation}[chapter]
\newtheorem{corollary}{\color{BrickRed} Corollary}[chapter]

\newtheorem{equality}{\color{BrickRed} Equality}[chapter]
\newtheorem{inequality}{\color{BrickRed} Inequality}[chapter]

\theoremstyle{definition}
\newtheorem{definition}{\color{BrickRed} Definition}[chapter]
\newtheorem{example}{\color{BrickRed} Example}[chapter]
\newtheorem{question}{\color{BrickRed} Question}[chapter]
\newtheorem{exercise}{\color{BrickRed} Exercise}[chapter]

\theoremstyle{remark}
\newtheorem*{comment}{Comment}
\AtBeginEnvironment{comment}{\color{gray}}

% ===============================================
%   Term Index
% ===============================================

\makeindex
\NewDocumentCommand{\defterm}{som}{%
  {\it #3}%
  \IfBooleanTF{#1}{}{\index{\IfNoValueTF{#2}{#3}{#2}}}%
}
\NewDocumentCommand{\keyterm}{som}{%
  {\bf #3}%
  \IfBooleanTF{#1}{}{\index{\IfNoValueTF{#2}{#3}{#2}}}%
}

% ===============================================
%   Math Symbol
% ===============================================

\newcommand{\argmin}{\operatornamewithlimits{argmin}}
