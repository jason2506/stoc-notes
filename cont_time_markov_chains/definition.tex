\section{Definition}

\begin{definition}
The collection $ \mathbb{X} = \{ X(t) \mid t \ge 0 \} $ of non-negative integral random variable is a \defterm{continuous-time Markov chain} if
\[ P(X(s + t) = j \mid X(s) = i \wedge \forall(0 \le r < s) .X(r) = x(r)) = P(X(s + t) = j \mid X(s) = i) \]
holds for any $ s $, $ t $, $ i $, $ j $, and non-negative integral function $ x(\cdot) $.
\begin{comment}
Non-homogeneous Poisson process with intensity function $ \lambda(\cdot) $ is a continuous-time Markov chain given $ N(s) = n $, the number of events in time $ [s, s + t] $ is a Poisson distribution with parameter
\[ \int_{s}^{s + t} \lambda(x) \mathrm{d}x, \]
having nothing to do with $ N(s - \epsilon) $ for all $ \epsilon \ge 0 $.
\end{comment}
\end{definition}

\begin{definition}
If the conditional probability $ P(X(s + t) = j \mid X(s) = i) $ is independent of $ s $ for all states $ i $ and $ j $, then the continuous-time Markov chain is said to have \defterm{homogeneous transition} probabilities. (我們只會討論這一類。)
\end{definition}

\begin{example}
Homogeneous Poisson process 就是一個 continuous-time Markov chain with homogeneous transition probabilities.
\end{example}

\begin{observation}
The time for $ \mathbb{X} $ to stay in state $ i $ is an exponential distribution.
\begin{proof}
Suppose that $ X(r) = i $ for some $ r \ge 0 $.
Let $ T_{i} $ be the waiting time of transitioning to a different state. Then,
\begin{eqnarray*}
P(T_{i} \ge s + t \mid T_{i} \ge s)
  & = & P(\forall(r + s \le v \le r + s + t).X(v) = i \mid \forall(r \le u \le r + s).X(u) = i) \\
  & = & P(\forall(r + s \le v \le r + s + t).X(v) = i \mid X(r + s) = i) \\
  & = & P(\forall(r \le v \le r + t).X(v) = i \mid X(r) = i) \\
  & = & P(T_{i} \ge t).
\end{eqnarray*}
\end{proof}
\begin{comment}
\begin{alignat*}{3}
  & P(T_{i} \ge t)
    & \quad=\quad & P(T_{i} \ge s + t \mid T_{i} \ge s) \\
  & & \quad=\quad & \frac{P(T_{i} \ge s + t, T_{i} \ge s)}{P(T_{i} \ge s)} \\
  & & \quad=\quad & \frac{P(T_{i} \ge s + t)}{P(T_{i} \ge s)}, \\
\Rightarrow\quad
  & P(T_{i} \ge s + t)
    & \quad=\quad & P(T_{i} \ge s) \cdot P(T_{i} \ge t).
\end{alignat*}
It is satisfied when $ X $ is exponentially distributed (for $ e^{-\lambda_{i}(s + t)} = e^{-\lambda_{i} s} \cdot e^{-\lambda_{i} t} $).

我們稱這個 exponential distribution 的參數為 $ \lambda_{i} $。
\end{comment}
\end{observation}

\begin{observation}
Given $ X(s) = i $, the probability that ``the next state other than $ i $ is $ j $'' is a value $ \mathbb{P}[i, j] $ that is independent of $ s $.
\begin{proof}
Let $ \mathbb{T} $ be the waiting time of the first transition from state $ i $ to a state other than $ i $. We have
\begin{eqnarray*}
\mathbb{P}[i, j]
  & = & P(X(T_{i} + s) = j \mid X(s) = i) \\
  & = & P(X(T_{i}) = j \mid X(0) = i).
\end{eqnarray*}

Therefore, $ \mathbb{X} $ can be completely characteried by
\begin{enumerate}
  \item The exponential time with parameters $ \lambda_{i} $ for all state $ i \in \mathcal{S} $.
  \item The transition probability $ \mathbb{P}[i, j] $ of all $ i, j \in \mathcal{S} $.
\end{enumerate}
\end{proof}
\end{observation}

\begin{example} \label{ex:barber_shop}
理髮店具有三種狀態:沒人(state 0)、理髮(state 1)、洗頭(state 2)。每位客人來到店內後,必定會先剪完頭髮,然後洗完頭才會離開。若店內已經有客人,新來的客人會立刻離開。若客人來的過程為 Poisson process with rate $ \lambda_{0} $,且理髮與洗頭所花費的時間分別為 exponential distribution with $ \lambda_{1} $ 與 $ \lambda_{2} $,則店的狀態改變過程就是一個 continuous-time Markov chain,且
\[ \mathbb{P}[0, 1] = \mathbb{P}[1, 2] = \mathbb{P}[2, 0] = 1. \]
\end{example}
