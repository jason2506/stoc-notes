\section{Branching Processes}

\begin{question}
一開始有 X(0) 個 life forms,每個 life form 以 $ p_{i} $ 的機率產生 $ i $($ i = 0, \cdots, m $)個後代,然後死去。
What's the probability of extinction?

\begin{description}
  \item[Case 1] $ p_{0} = 0 \Rightarrow $ 不會 extinction
  \item[Case 2] $ p_{0} > 0 $,

    Let $ X(t) $ be the population size of $ t $-th generation, $ \mathcal{S} = \{ 0, 1, \cdots \} $.
    \begin{itemize}
      \item State 0 is recurrent, other states are transient.
      \item 所以不是 extinct 就是 $ \lim_{t \to \infty} X(t) = \infty $。
      \item 假設 $ X(n - 1) = i $,令 $ Z_{k} $($ k = 1, \cdots, i $)為 $ k $-th lift form 在 $ n $-th generation 的後代數。 \\
        $ \Rightarrow Z_{1}, \cdots, Z_{i} $ i.i.d. with $ \mu = \sum_{j = 1}^{m} j \cdot p_{j} $.
        \begin{eqnarray*}
          & & E[X(n) \mid X(n - 1) = i] = i \cdot \mu \\
          & \Rightarrow & E[X(n) \mid X(n - 1)] = X(n - 1) \cdot \mu \\
          & \Rightarrow & E[X(n)] = E[E[X(n) \mid X(n - 1)]] = E[X(n - 1)] \cdot \mu = X(0) \cdot \mu^{n}
        \end{eqnarray*}
      \item Probability of extinction $ e_{i} = P(\text{extinction} \mid X(0) = i) $.
        \begin{description}
          \item[Case 2-1] $ \mu < 1 $,
            \begin{eqnarray*}
              1 - e_{i}
                & = & \lim_{n \to \infty} P(X(n) \ge 1 \mid X(0) = i) \\
                & = & \lim_{n \to \infty} \sum_{j = 1}^{\infty} P(X(n) = j \mid X(0) = i) \\
                & \le & \lim_{n \to \infty} \sum_{j = 1}^{\infty} j \cdot P(X(n) = j \mid X(0) = i) \\
                & = & \lim_{n \to \infty} E[X(n) \mid X(0) = i] \\
                & = & \lim_{n \to \infty} i \cdot \mu^{n} \\
                & = & 0.
            \end{eqnarray*}
          \item[Case 2-2] $ \mu \ge 1 $,
            \begin{eqnarray*}
              e_{i}
                & = & \sum_{j = 0}^{m} P(\text{extinction} \mid X(1) = j) \cdot \mathbb{P}[i, j] \\
                & = & \sum_{j = 0}^{m} e_{j} \cdot p_{j} \\
                & = & \sum_{j = 0}^{m} e_{1}^{j} \cdot p_{j}.
            \end{eqnarray*}
        \end{description}
    \end{itemize}
\end{description}
\end{question}

\begin{example}
\begin{eqnarray*}
  & & p_{0} = 0.5, p_{1} = p_{2} = 0.25 \\
  & \Rightarrow & \mu = 1 \cdot 0.25 + 2 \cdot 0.25 = 0.75 < 1 \\
  & \Rightarrow & e_{i} = 1 \text{ for all } i.
\end{eqnarray*}
\end{example}

\begin{example}
\begin{eqnarray*}
  & & p_{0} = p_{1} = 0.5, p_{2} = 0.25 \\
  & \Rightarrow & \mu = 1 \cdot 0.25 + 2 \cdot 0.5 = 1.25 \\
  & \Rightarrow & e_{1} = 0.25 + 0.25 \cdot e_{1} + 0.5 \cdot e_{1}^{2} \\
  & \Rightarrow & e_{1} \in \{ 0.5, 1 \}.
\end{eqnarray*}

Since $ \lim_{n \to \infty} E[X(n)] = \lim_{n \to \infty} X(0) \cdot \mu^{n} = \infty $, we know that $ e_{1} \neq 1 $ and hence $ e_{1} = 0.5 $.
\end{example}
