\section{Some Markov Chains}

% ===============================================
%   Infinite States
% ===============================================
\subsection{Infinite States} \label{subsec:infinite-states}

\begin{figure}[htp]
\centering
\begin{tikzpicture}
  \node (X-3) {};
  \foreach \x in {-2,...,2}{
    \pgfmathtruncatemacro{\y}{\x - 1}
    \node[state] (X\x) [right of=X\y] {$ \x $};
  }
  \node (X3) [right of=X2] {};

  \path[-] (X-3) edge [loosely dotted,line width=.15em] (X-2)
           (X3)  edge [loosely dotted,line width=.15em] (X2);
  \foreach \x in {-1,...,2}{
    \pgfmathtruncatemacro{\y}{\x - 1}
    \path[-] (X\y) edge [bend right=0] (X\x);
  }
\end{tikzpicture}
\[ \mathbb{P}[i + 1, i] = \mathbb{P}[i - 1, i] = 0.5 \]
\end{figure}

\begin{question}
The above Markov chain $ \mathbb{X} $ is recurrent or transient?
\begin{eqnarray*}
f_{0}
  & = & P(\exists t \ge 1 . X(t) = 0 \mid X(0) = 0) \\
  & = & P(\exists t \ge {\color{red} 2} . X(t) = 0 \mid X(1) = 1) \cdot 0.5
        + P(\exists t \ge {\color{red} 2} . X(t) = 0 \mid X(1) = -1) \cdot 0.5 \\
  & = & P(\exists t \ge 2 . X(t) = 0 \mid X(1) = -1) \\
  & = & P(\exists t \ge {\color{red} 1} . X(t) = 0 \mid X({\color{red} 0}) = -1) \\
  & = & P(\exists t \ge 1 . X(t) = {\color{red} 1} \mid X(0) = {\color{red} 0}) \\
  & = & \underbrace{P(\exists t \ge 1 . X(t) = 1 \mid X(1) = 1)}_{= 1} \cdot 0.5
        + P(\exists t \ge 1 . X(t) = 1 \mid X(1) = -1) \cdot 0.5 \\
  & = & 0.5 + P(\exists t \ge t' . X(t) = 1 \mid X(t') = 0) \cdot P(\exists t' \ge 1 . X(t') = 0 \mid X(1) = -1) \cdot 0.5 \\
  & = & 0.5 + f_{0} \cdot f_{0} \cdot 0.5.
\end{eqnarray*}
Thus, $ f_{0} = \frac{1}{2} + \frac{f_{0}^{2}}{2} \Rightarrow (f_{0} - 1)^{2} = 0 \Rightarrow f_{0} = 1 $.
\end{question}

\begin{exercise}
The Markov chain $ \mathbb{X} $ with infinite states is transient if $ \mathbb{P}[i + 1, i] \neq 0.5 $ for all state $ i $. Why?
\end{exercise}

% ===============================================
%   Gambler's Ruin
% ===============================================
\subsection{Gambler's Ruin} \label{subsec:gambler-ruin}

\begin{figure}[htp]
\centering
\begin{tikzpicture}
  \node[state] (Xi-1)                             {$ i - 1 $};
  \node[state] (Xi)   [right of=Xi-1]             {$ i $};
  \node[state] (Xi+1) [right of=Xi]               {$ i + 1 $};
  \node[state] (X0)   [left  of=Xi-1,xshift=-2em] {$ 0 $};
  \node[state] (Xm)   [right of=Xi+1,xshift=2em]  {$ m $};

  \path (X0) edge [loop left] node {1} (X0)
        (Xi) edge [bend right=40] node [swap] {$1 - q$} (Xi-1)
        (Xi) edge [bend left=40] node {$ q $} (Xi+1)
        (Xm) edge [loop right] node {1} (Xm);
  \path[-] (X0) edge [loosely dotted,line width=.15em] (Xi-1)
           (Xm) edge [loosely dotted,line width=.15em] (Xi+1);
\end{tikzpicture}

States $ 0 $ and $ m $ are recurrent, other states are transient.
\end{figure}

\begin{question}
What's $ p_{i} = P(\exists \tau \ge 0. \forall t \ge \tau. X(t) = m \mid X(0) = i) $?

For $ i = 1, \cdots, m - 1 $,
\begin{eqnarray*}
p_{i}
  & = & q \cdot p_{i + 1} + (1 - q) \cdot p_{i - 1} \\
  & = & q \cdot p_{i} + (1 - q) \cdot p_{i} \\
  & \Rightarrow & p_{i + 1} - p_{i} = r \cdot (p_{i} - p_{i - 1}), \text{ where } r = \frac{1 - q}{q} \\
  & \Rightarrow & p_{j} = \frac{s_{j - 1}}{s_{m - 1}}, \text{ where } s_{j} = \sum_{k = 0}^{j} r^{k}.
\end{eqnarray*}

{\bf Special Case:} $ q = \frac{1}{2} \Rightarrow r = 1 \Rightarrow s_{j} = j + 1 \Rightarrow p_{j} = \frac{j}{m} $.
\end{question}

% ===============================================
%   Gambler's Ruin: A Special Case
% ===============================================
\subsection{Gambler's Ruin: A Special Case}

\begin{figure}[htp]
\centering
\begin{tikzpicture}
  \node[state] (Xi-1)                             {$ i - 1 $};
  \node[state] (Xi)   [right of=Xi-1]             {$ i $};
  \node[state] (Xi+1) [right of=Xi]               {$ i + 1 $};
  \node[state] (X0)   [left  of=Xi-1,xshift=-2em] {$ 0 $};
  \node        (Xm)   [right of=Xi+1,xshift=2em]  {};

  \path (X0) edge [loop left] node {1} (X0)
        (Xi) edge [bend right=40] node [swap] {$1 - q$} (Xi-1)
        (Xi) edge [bend left=40] node {$ q $} (Xi+1);
  \path[-] (X0) edge [loosely dotted,line width=.15em] (Xi-1)
           (Xm) edge [loosely dotted,line width=.15em] (Xi+1);
\end{tikzpicture}

A special case of \nameref{subsec:gambler-ruin}, $ \mathbb{X}_{m} $, with $ m \to \infty $.
\end{figure}

\noindent Let $ p_{i} = P(\exists t \ge 0. X(t) = 0 \mid X(0) = i) $.
\begin{description}
  \item[Case 1.] $ q = 0.5 $:
    \begin{eqnarray*}
    p_{i}
      & = & \lim_{m \to \infty} P(\exists \tau \ge 0. \forall t \ge \tau. {\color{red} X_{m}}(t) = 0 \mid {\color{red} X_{m}}(0) = i) \\
      & = & \lim_{m \to \infty} 1 - \frac{i}{m} \\
      & = & 1.
    \end{eqnarray*}
  \item[Case 2.] $ q > 0.5 $:
    \begin{eqnarray*}
    p_{i}
      & = & \lim_{m \to \infty} \frac{s_{i - 1}}{s_{m - 1}} \\
      & = & \lim_{m \to \infty} 1 - \frac{(1 - r^{i}) / (1 - r)}{(1 - r^{m}) / (1 - r)} \\
      & = & r^{i}.
    \end{eqnarray*}
  \item[Case 3.] $ q < 0.5 $: $ p_{i} = 1 $.
\end{description}

\begin{exercise}
Show that $ p_{i} = 1 $ if $ q < 0.5 $.
\end{exercise}

% ===============================================
%   Invincible Gambler
% ===============================================
\subsection{Invincible Gambler} \label{subsec:invincible-gambler}

\begin{figure}[htp]
\centering
\begin{tikzpicture}
  \node[state] (Xi-1)                             {$ i - 1 $};
  \node[state] (Xi)   [right of=Xi-1]             {$ i $};
  \node[state] (Xi+1) [right of=Xi]               {$ i + 1 $};
  \node[state] (X1)   [left  of=Xi-1,xshift=-2em] {$ 1 $};
  \node[state] (X0)   [left  of=X1]               {$ 0 $};
  \node[state] (Xm)   [right of=Xi+1,xshift=2em]  {$ m $};

  \path (X0) edge [bend right=40] node [swap] {1} (X1)
        (X1) edge [bend right=40] node [swap] {$1 - q$} (X0)
        (Xi) edge [bend right=40] node [swap] {$1 - q$} (Xi-1)
        (Xi) edge [bend left=40] node {$ q $} (Xi+1)
        (Xm) edge [loop right] node {1} (Xm);
  \path[-] (X1) edge [loosely dotted,line width=.15em] (Xi-1)
           (Xm) edge [loosely dotted,line width=.15em] (Xi+1);
\end{tikzpicture}

State $ m $ is recurrent, other states are transient.
\end{figure}

\begin{question}
What's the excepted number of steps to reach $ m $ from $ 0 $?
\begin{description}
  \item[Case 1.] $ q = 0.5 $: $ m^{2} $
  \item[Case 2.] $ q \neq 0.5 $:
    \[ 1 + \frac{2 \cdot r^{m + 1} - (m + 1) \cdot r^{2} + m - 1}{(1 - r)^{2}} \]
    \begin{alignat*}{3}
    q < 0.5 & \quad\Rightarrow\quad & r > 1 & \quad\Rightarrow\quad & \text{exp in $ m $} \\
    q > 0.5 & \quad\Rightarrow\quad & r < 1 & \quad\Rightarrow\quad & \text{linear in $ m $}
    \end{alignat*}
\end{description}
\end{question}

\begin{example}
$ k $-CNF (Conjunction Normal Form) is a conjunction of clauses, where a clause is a disjunction of $ k $ literals. The following expression is an example of 3-CNF:
\[ (x_{1} \vee \underbrace{\neg x_{2}}_{\text{literal}} \vee \underbrace{x_{3}}_{\text{variable}}) \wedge \underbrace{(x_{2} \vee \neg x_{3} \vee x_{5})}_{\text{clause}} \wedge \cdots \wedge (x_{7} \vee \neg x_{6} \vee \neg x_{4}) \]

The goal of $ k $-SAT ($ k $-satisfiability) problem is to find a truth assignment, which makes all clauses of the $ k $-CNF expression $ \Phi $ be true.

$ k $-SAT 的隨機演算法挑選一個為 {\tt false} 的 clause,並從其 $ k $ 個 variable 中挑選一個反轉。這種演算法可以類比成 \nameref{subsec:invincible-gambler},$ X(t) $ 代表 current solution $ A $ 與 true solution $ A^{*} $ 有幾個 variable 的值相同。

因此,當 $ k \le 2 $ 時,複雜度的期望值為 polynomial time;當 $ k \ge 3 $ 時,複雜度的期望值為 exponential time。
\end{example}
