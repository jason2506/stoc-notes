\section{Convergence of Markov chain}

\begin{definition}
If $ \pi_{j} = \lim_{n \to \infty} \mathbb{P}^{n} [i, j] $ holds for all states $ i \in \mathcal{S} $, then $ \pi_{j} $ is the \defterm{limiting probability} of state $ j $.

\begin{comment}
If all states $ i $ have limiting probability, then
\[ \lim_{n \to \infty} \mathbb{P}^{n} [i, j] =
\begin{bmatrix}
  \boldsymbol\pi \\ \vdots \\ \boldsymbol\pi
\end{bmatrix}, \]
where $ \boldsymbol\pi $ is the row vector $ (\pi_{1}, \pi_{2}, \cdots) $.
\end{comment}
\end{definition}

\begin{definition}
Non-negative row vector $ \boldsymbol\pi = (\pi_{1}, \pi_{2}, \cdots) $ with $ \sum_{i \in \mathcal{S}} \pi_{i} = 1 $ is called a \defterm{stationary distribution} or \defterm{steady-state distribution} of $ \mathbb{X} $ if $ \boldsymbol\pi \mathbb{P} = \boldsymbol\pi $ (normalized left eignvector with eignvalue $ 1 $).

\begin{comment}
Why ``steady-state''?

If $ P(X(0) = i) = \pi_{i} $ for all $ i \in \mathcal{S} $, then
\begin{eqnarray*}
P(X(1) = i)
  & = & \sum_{i \in \mathcal{S}} P(X(1) = j \mid X(0) = i) \cdot P(X(0) = i) \\
  & = & \sum_{i \in \mathcal{S}} \pi_{i} \cdot \mathbb{P}[i, j] \\
  & = & \pi_{j},
\end{eqnarray*}
for all $ j \in \mathcal{S} $.
\end{comment}
\end{definition}

\begin{definition}
$ \mu_{i} = \sum_{n \ge 1} n \cdot f_{i}^{(n)} $ is the \defterm{excepted return time} of state $ i $, where 
\[ f_{i}^{(n)} = P \left( n = \argmin_{m \ge 1} ( X(m) = i ) \;\middle|\; X(0) = i \right). \]

\begin{itemize}
  \item state $ i $ is \defterm{positive recurrent} if $ \mu_{i} < \infty $,
  \item state $ i $ is \defterm{null recurrent} if $ \mu_{i} = \infty $,
  \item $ \mathbb{X} $ is \defterm*{positive recurrent} if all states of $ \mathbb{X} $ are positive recurrent,
  \item $ \mathbb{X} $ is \defterm*{null recurrent} if all states of $ \mathbb{X} $ are null recurrent.
\end{itemize}

\begin{comment}
\[ f_{i} = \sum_{n = 0}^{\infty} f_{i}^{(n)} = \text{recurrent probability of state $ i $}. \]
\end{comment}
\end{definition}

\begin{exercise}
Proof that $ i \leftrightarrow j \wedge i \text{ is positive recurrent } \Rightarrow j \text{ is positive recurrent}. $
\end{exercise}

\begin{definition}
The \defterm{period} of state $ i $ is the greatest common divisor (gcd) of all numbers $ n \ge 1 $ with $ \mathbb{P}[i, i] > 0 $.

\begin{itemize}
  \item state $ i $ is \defterm{aperiodic} if the period of $ i $ is $ 1 $,
  \item $ \mathbb{X} $ is \defterm*{aperiodic} if all states of $ \mathbb{X} $ are aperiodic.
\end{itemize}
\end{definition}

\begin{lemma} \label{lem:lim-p_nij-exists}
If state $ j $ is aperiodic and positive recurrent, then $ \lim_{n \to \infty} \mathbb{P}^{n}[i, j] $ exists, is positive and is independent of $ i $ for all states $ i $ with $ i \leftrightarrow j $ (唯一不證的 lemma).

\begin{comment}
In general, if $ d $ is the period of $ j $, then
\[ \lim_{t \to \infty} \mathbb{P}^{td}[i, j] = \frac{d}{\mu_{i}} \]
\end{comment}
\end{lemma}

\begin{inequality} \label{ineq:sum-pi-le-1}
If the limiting probability $ \pi_{i} $ exists for all $ i \in \mathcal{S} $, then $ \sum_{i \in \mathcal{S}} \pi_{i} \le 1 $.

\begin{proof}
\begin{eqnarray*}
\sum_{j = 1}^{m} \pi_{j}
  & = & \sum_{j = 1}^{m} \lim_{n \to \infty} \mathbb{P}^{n}[i, j] \\
  & = & \lim_{n \to \infty} \sum_{j = 1}^{m} \mathbb{P}^{n}[i, j] \\
  & \le & \lim_{n \to \infty} \sum_{j = 1}^{\infty} \mathbb{P}^{n}[i, j] \\
  & = & 1,
\end{eqnarray*}
and hence
\[ \sum_{i \in \mathcal{S}} \pi_{i} = \lim_{m \to \infty} \sum_{j = 1}^{m} \pi_{j} \le 1. \]
\end{proof}
\end{inequality}

\begin{inequality} \label{ineq:pi-ge-pi-times-p}
If the limiting probability $ \pi_{i} $ exists for all $ i \in \mathcal{S} $, then $ \pi_{i} \ge \sum_{i \in \mathcal{S}} \pi_{i} \cdot \mathbb{P}[i, j] $ holds for all state $ j \in \mathcal{S} $.

\begin{proof}
For all $ m, n \ge 1 $,
\begin{alignat*}{4}
  & \mathbb{P}^{n + 1}[i, j]
    & \quad=\quad & \sum_{k \in \mathcal{S}} \mathbb{P}^{n}[i, k] \cdot \mathbb{P}[k, j] \\
  & & \quad\ge\quad & \sum_{k = 0}^{m} \mathbb{P}^{n}[i, k] \cdot \mathbb{P}[k, j]. \\
\Rightarrow\quad
  & \pi_{j}
    & \quad=\quad & \lim_{n \to \infty} \mathbb{P}^{n + 1}[i, j] \\
  & & \quad\ge\quad & \lim_{n \to \infty} \sum_{k = 0}^{m} \mathbb{P}^{n}[i, k] \cdot \mathbb{P}[k, j]. \\
  & & \quad=\quad & \sum_{k = 0}^{m} \pi_{k} \cdot \mathbb{P}[k, j]. \\
\Rightarrow\quad
  & \underbrace{\lim_{m \to \infty} \pi_{j}}_{= \pi_{j}}
    & \quad\ge\quad & \lim_{m \to \infty} \sum_{k = 0}^{m} \pi_{k} \cdot \mathbb{P}[k, j]. \\
\end{alignat*}
\end{proof}
\end{inequality}

\begin{equality} \label{eq:pi-eq-pi-times-p}
If the limiting probability $ \pi_{j} $ exists for all $ j \in \mathcal{S} $, then $ \boldsymbol\pi = \boldsymbol\pi \mathbb{P} $.

\begin{proof}
Assume that $ \boldsymbol\pi \neq \boldsymbol\pi \mathbb{P} $,
\begin{eqnarray*}
\sum_{j \in \mathcal{S}} \pi_{j}
  & > & \sum_{j \in \mathcal{S}} \sum_{i \in \mathcal{S}} \pi_{i} \cdot \mathbb{P}[i, j]
    \text{\hspace{1em} \small(by \autoref{ineq:pi-ge-pi-times-p})} \\
  & = & \sum_{i \in \mathcal{S}} \pi_{i} \underbrace{\sum_{j \in \mathcal{S}} \mathbb{P}[i, j]}_{= 1} \\
  & = & \sum_{i \in \mathcal{S}} \pi_{i},
\end{eqnarray*}
contradiction.
\end{proof}
\end{equality}

\begin{inequality} \label{ineq:p-ge-pi}
If the limiting probability $ \pi_{i} $ exists for all $ i \in \mathcal{S} $ and $ \mathbf{p} $ is a stationary distribution of $ \mathbb{X} $, then $ p_{i} \ge \pi_{i} $ holds for all $ i \in \mathcal{S} $.

\begin{proof}
Assume that $ p $ is the distribution of $ X(0) $,
\begin{alignat*}{4}
  & p_{j}
    & \quad=\quad & P(X(n) = j) \\
  & & \quad=\quad & \sum_{i \in \mathcal{S}} \mathbb{P}^{n}[i, j] \cdot p_{i} \\
  & & \quad\ge\quad & \sum_{i = 1}^{m} \mathbb{P}^{n}[i, j] \cdot p_{i}. \\
\Rightarrow\quad
  & \lim_{n \to \infty} p_{j}
    & \quad\ge\quad & \lim_{n \to \infty} \sum_{i = 1}^{m} \mathbb{P}^{n}[i, j] \cdot p_{i} \\
  & & \quad=\quad & \sum_{i = 1}^{m} \pi_{j} \cdot p_{i}. \\
\Rightarrow\quad
  & \underbrace{\lim_{m \to \infty} \lim_{n \to \infty} p_{j}}_{= p_{j}}
    & \quad\ge\quad & \lim_{m \to \infty} \sum_{i = 1}^{m} \pi_{j} \cdot p_{i} \\
  & & \quad=\quad & \sum_{i \in \mathcal{S}} \pi_{j} \cdot p_{i} \\
  & & \quad=\quad & \pi_{j}.
\end{alignat*}
\end{proof}
\end{inequality}

\begin{inequality} \label{ineq:p-le-pi}
If the limiting probability $ \pi_{i} $ exists for all $ i \in \mathcal{S} $ and $ \mathbf{p} $ is a stationary distribution of $ \mathbb{X} $, then $ p_{i} \le \pi_{i} $ holds for all $ i \in \mathcal{S} $.

\begin{proof}
Assume that $ p $ is the distribution of $ X(0) $,
\begin{alignat*}{4}
  & p_{j}
    & \quad=\quad & P(X(n) = j) \\
  & & \quad=\quad & \sum_{i \in \mathcal{S}} \mathbb{P}^{n}[i, j] \cdot p_{i} \\
  & & \quad\le\quad & \sum_{i = 1}^{m} \mathbb{P}^{n}[i, j] \cdot p_{i} + \sum_{i > m} p_{i}. \\
\Rightarrow\quad
  & \lim_{n \to \infty} p_{j}
    & \quad\le\quad & \lim_{n \to \infty} \sum_{i = 1}^{m} \mathbb{P}^{n}[i, j] \cdot p_{i} + \sum_{i > m} p_{i} \\
  & & \quad=\quad & \sum_{i = 1}^{m} \pi_{j} \cdot p_{i} + \sum_{i > m} p_{i}. \\
\Rightarrow\quad
  & \underbrace{\lim_{m \to \infty} \lim_{n \to \infty} p_{j}}_{= p_{j}}
    & \quad\le\quad & \lim_{m \to \infty} \sum_{i = 1}^{m} \pi_{j} \cdot p_{i} + \sum_{i > m} p_{i} \\
  & & \quad=\quad & \sum_{i \in \mathcal{S}} \pi_{j} \cdot p_{i} + 0 \\
  & & \quad=\quad & \pi_{j}.
\end{alignat*}
\end{proof}
\end{inequality}

\begin{theorem}
If $ \mathbb{X} $ is irreducible, positive recurrent, and aperiodic, then
\begin{enumerate}
  \item the limiting probability $ \pi_{i} $ of each state $ i $ exists, and
  \item $ \boldsymbol\pi = (\pi_{1}, \pi_{2}, \cdots) $ is the unique stationary distribution of $ \mathbb{X} $.
\end{enumerate}

\begin{proof}[proof (exists)]
By \autoref{lem:lim-p_nij-exists}, \autoref{ineq:sum-pi-le-1}, and \autoref{eq:pi-eq-pi-times-p}, we know that
\[ \frac{\boldsymbol\pi}{\sum_{j \in \mathcal{S}} \pi_{j}} \]
is a stationary distribution.
\end{proof}

\begin{comment}
By \autoref{lem:lim-p_nij-exists}, we know that the limiting probability $ \pi_{i} $ exists for all $ i \in \mathcal{S} $,
\[ \sum_{j \in \mathcal{S}} \pi_{j} \neq \infty, \text{\hspace{1em} \small (by \autoref{ineq:sum-pi-le-1})} \]
and
\[ \frac{\boldsymbol\pi}{\sum_{j \in \mathcal{S}} \pi_{j}} = \frac{\boldsymbol\pi \mathbb{P}}{\sum_{j \in \mathcal{S}} \pi_{j}}. \text{\hspace{1em} \small (by \autoref{eq:pi-eq-pi-times-p})} \]
\end{comment}

\begin{proof}[proof (unique)]
By \autoref{ineq:p-ge-pi} and \autoref{ineq:p-le-pi}, we know that the stationary distribution $ p $, which is equal to the limiting probability $ \pi_{i} $, is the unique stationary distribution of $ \mathbb{X} $.
\end{proof}
\end{theorem}

\begin{example}
Say that the matrix of transition probabilities of $ \mathbb{X} $ is
\[ \mathbb{P} = \begin{bmatrix}
  \alpha  & 1 - \alpha \\
  \beta   & 1 - \beta
\end{bmatrix}, 0 < \alpha, \beta < 1. \]
Since $ \mathbb{X} $ is irreducible, possitive recurrent, and aperiodic {\color{red} (Why?)}, we know that $ \boldsymbol\pi \mathbb{P} = \boldsymbol\pi $, and hence
\begin{eqnarray*}
& & (x, 1 - x) \cdot \begin{bmatrix}
  \alpha  & 1 - \alpha \\
  \beta   & 1 - \beta
\end{bmatrix} = (x, 1 - x) \\
& \Rightarrow & \boldsymbol\pi = \left( \frac{\beta}{1 + \beta - \alpha}, \frac{1 - \alpha}{1 + \beta - \alpha} \right)
\end{eqnarray*}
\end{example}

\begin{definition}
$ r_{j} $ is the \defterm{long-run proportion} of state $ j $ if
\[ r_{j} = \lim_{n \to \infty} \frac{1}{n} \sum_{t = 1}^{n} \mathbb{P}^{t}[i, j] \]
holds for all states $ i \in \mathcal{S} $.

\begin{comment}
Why this name?

Support $ N $ is a uniform distribution over $ \{ 1, \cdots, n \} $. The ``long-run proportion of state $ j $'' is
\begin{eqnarray*}
  &   & \lim_{n \to \infty} P(X(N) = j \mid X(0) = i) \\
  & = & \lim_{n \to \infty} P(X(N) = j \mid N = t, X(0) = i) \cdot P(N = t \mid X(0) = i) \\
  & = & \lim_{n \to \infty} P(X(N) = j \mid X(0) = i) \cdot \frac{1}{n} \\
  & = & \lim_{n \to \infty} \frac{1}{n} \sum_{t = 1}^{n} \mathbb{P}^{t}[i, j]
\end{eqnarray*}

(Proportion:state 在 t 時間內出現的比例有多高。)
\end{comment}
\end{definition}

\begin{lemma} \label{lem:r-eq-rp}
If the long-run proportions $ r_{j} $ of all states $ j $ exist, then $ \mathbf{r} = (r_{1}, r_{2}, \cdots) = \mathbf{r}\mathbb{P} $.

\begin{proof}
Let $ \mathbb{R} = [\mathbf{r}^{\intercal}, \cdots, \mathbf{r}^{\intercal}]^{\intercal} = \lim_{n \to \infty} \frac{1}{n} \sum_{t = 1}^{n} \mathbb{P}^{t} $,
\begin{eqnarray*}
\mathbb{R}\mathbb{P}
  & {\color{red} =} & \lim_{n \to \infty} \frac{1}{n} \sum_{t = 1}^{n} \mathbb{P}^{t + 1} \\
  & = & \lim_{n \to \infty} \frac{1}{n} \sum_{t = 1}^{n} \mathbb{P}^{t}
      + \underbrace{\lim_{n \to \infty} \frac{1}{n} \sum_{t = 1}^{n} (\mathbb{P}^{t + 1} - \mathbb{P}^{t})}_{= 0} \\
  & = & \mathbb{R}.
\end{eqnarray*}
\end{proof}
\end{lemma}

\begin{lemma} \label{lem:pi-eq-r}
If $ \boldsymbol\pi $ is a stationary distribution of $ \mathbb{X} $, and $ \mathbf{r} $ is a long-run proportion of $ \mathbb{X} $, then $ \boldsymbol\pi = \mathbf{r} $.

\begin{proof}
Let $ \mathbb{R} = [\mathbf{r}^{\intercal}, \cdots, \mathbf{r}^{\intercal}]^{\intercal} = \lim_{n \to \infty} \frac{1}{n} \sum_{t = 1}^{n} \mathbb{P}^{t} $,
\begin{eqnarray*}
\mathbf{r}
  & = & \boldsymbol\pi \mathbb{R}
    \text{ \hspace{1em}\small(since $ \textstyle\sum_{k \in \mathcal{S}} \pi_{k} = 1 $)} \\
  & = & \boldsymbol\pi \cdot \left( \lim_{n \to \infty} \frac{1}{n} \sum_{t = 1}^{n} \mathbb{P}^{t} \right) \\
  & {\color{red} =} & \lim_{n \to \infty} \frac{1}{n} \sum_{t = 1}^{n} \boldsymbol\pi \mathbb{P}^{t} \\
  & = & \lim_{n \to \infty} \frac{1}{n} \sum_{t = 1}^{n} \boldsymbol\pi \\
  & = & \boldsymbol\pi.
\end{eqnarray*}
\end{proof}
\end{lemma}

\begin{theorem}
If $ \mathbb{X} $ is irreducible (and the long-run proportions $ r_{j} $ of all states $ j $ exist), and $ \sum_{j \in \mathcal{S}} r_{j} = 1 $, then $ \mathbf{r} = (r_{1}, r_{2}, \cdots) $ is the unique stationary distribution of $ \mathbb{X} $.

\begin{proof}
Immedate from \autoref{lem:r-eq-rp} and \autoref{lem:pi-eq-r}.
\end{proof}
\end{theorem}

\begin{example}
\[ \bordermatrix{
    & \text{大} & \text{小} \cr
  \text{大} & \frac{1}{4} & \frac{3}{4} \cr
  \text{小} & \frac{1}{5} & \frac{4}{5}
}, \mathbf{r} = \boldsymbol\pi = \left( \frac{4}{19}, \frac{15}{19} \right). \]
\end{example}

\begin{example}
Let $ \mathcal{G} = \{ g_{1}, g_{2}, \cdots \} $ be the set of good states, $ \mathcal{B} = \{ b_{1}, b_{2}, \cdots \} $ be the set of bad states, and $ \mathcal{S} = \mathcal{G} \cup \mathcal{B} $. If $ \mathbb{X} $ is irreducible, then
\begin{description}
  \item[Q1.] What's the break down (transition from good state to bad state) rate of $ \mathbb{X} $?
  \item[Q2.] What's the excepted time of staying in good states ($ \mu_{\mathcal{G}} $)?
  \item[Q3.] What's the excepted time of staying in bad states ($ \mu_{\mathcal{B}} $)?
\end{description}

\begin{description}
  \item[A1.] \[ \sum_{i \in \mathcal{G}} \sum_{j \in \mathcal{B}} r_{i} \cdot \mathbb{P}[i, j]. \]
  \item[A2 \& A3.]
    Let $ G_{t} $ be the length of the $ t $-th phase of consecution good states, and $ B_{t} $ be the length of the $ t $-th phase of consecution bad states. According to the law of large numbers (LLN),
    \[ P \left( \lim_{t \to \infty} \frac{1}{t} \sum_{\tau = 1}^{t} (G_\tau + B_\tau) = \mu_{\mathcal{G}} + \mu_{\mathcal{B}} \right) = 1, \]
    and hence
    \begin{equation}
    \sum_{i \in \mathcal{G}} \sum_{j \in \mathcal{B}} r_{i} \cdot \mathbb{P}[i, j] = \frac{1}{\mu_{\mathcal{G}} + \mu_{\mathcal{B}}}. \label{eqn:a1}
    \end{equation}

    The long-run proportion of good states is
    \begin{eqnarray}
    \sum_{i \in \mathcal{G}} r_{i}
      & = & \lim_{t \to \infty} \frac{\sum_{\tau = 1}^{t} G_{\tau} / t}{\sum_{\tau = 1}^{t} (G_{\tau} + B_{\tau}) / t} \nonumber \\
      & = & \frac{\mu_{\mathcal{G}}}{\mu_{\mathcal{G}} + \mu_{\mathcal{B}}}, \label{eqn:a2}
    \end{eqnarray}
    and the long-run proportion of bad states is
    \begin{eqnarray}
    \sum_{i \in \mathcal{B}} r_{i}
      & = & \lim_{t \to \infty} \frac{\sum_{\tau = 1}^{t} B_{\tau} / t}{\sum_{\tau = 1}^{t} (G_{\tau} + B_{\tau}) / t} \nonumber \\
      & = & \frac{\mu_{\mathcal{B}}}{\mu_{\mathcal{G}} + \mu_{\mathcal{B}}}. \label{eqn:a3}
    \end{eqnarray}

    By (\ref{eqn:a2}) / (\ref{eqn:a1}), we get
    \[ \mu_{\mathcal{G}} = \frac{\sum_{i \in \mathcal{G}} r_{i}}{\sum_{i \in \mathcal{G}} \sum_{j \in \mathcal{B}} r_{i} \cdot \mathbb{P}[i, j]}. \]

    By (\ref{eqn:a3}) / (\ref{eqn:a1}), we get
    \[ \mu_{\mathcal{B}} = \frac{\sum_{i \in \mathcal{B}} r_{i}}{\sum_{i \in \mathcal{G}} \sum_{j \in \mathcal{B}} r_{i} \cdot \mathbb{P}[i, j]}. \]
\end{description}
\end{example}

\begin{theorem} \label{thm:long-run}
If $ \mathbb{X} $ is irreducible, the long-run proportion $ r_{i} $ of each state $ i \in \mathcal{S} $ exists with probability $ 1 $. Specifically,
\begin{enumerate}
  \item If $ i $ is positive recurrent, then $ P \left( r_{i} = \frac{1}{\mu_{i}} \right) = 1 $.
  \item If $ i $ is null recurrent or transient, then $ P(r_{i} = 0) = 1 $.
\end{enumerate}

\begin{proof}
Support $ X(0) = i $. Let $ T_{k} $ be the number of transitions required for $ \mathbb{X} $ to go from the $ k $-th $ i $ to $ (k + 1) $-th $ i $. Therefore, $ T_{1}, T_{2}, \cdots $ are i.i.d. with expectation $ \mu_{i} $. According to LLN,
\[ P \Bigg( \underbrace{\lim_{k \to \infty} \frac{1}{k} \sum_{t = 1}^{k} T_{t}}_{= \frac{1}{r_{i}}} = \mu_{i} \Bigg) = 1. \]
Thus, we get $ P \left( r_{i} = \frac{1}{\mu_{i}} \right) = 1 $ holds for all recurrent (both positive recurrent and null recurrent) state $ i $.

As for the case that $ i $ is transient, we know that the excepted number of visiting $ i < \infty $. That is, $ P(r_{i} = 0) = 1 $.
\end{proof}
\end{theorem}

\begin{example}
If $ \mathbb{X} $ is finite and irreducible, then $ \mathbb{X} $ is positive recurrent with probability $ 1 $. Why?

If $ \mathbb{X} $ is null recurrent, then $ r_{i} = 0 $ for all states $ i $, which makes a contradiction.
\end{example}

\begin{example}
\nameref{subsec:infinite-states} is null recurrent.

If it is positive recurrent, then $ r_{i} > 0 $ for all $ i \in \mathcal{S} $ (by \autoref{thm:long-run}).
In addition, by \autoref{lem:r-eq-rp}, we know that $ \mathbf{r}\mathbb{P} = \mathbf{r} $, where $ \mathbf{r} = (\cdots, r_{-1}, r_{0}, r_{1}, \cdots) $. Thus, $ \frac{1}{2} (r_{i - 1} + r_{i + 1}) = r_{i} $ for all $ i \in \mathcal{S} $. That is, $ \mathbf{r} $ is an infinite arithmetic progression, and $ \sum_{i \in \mathcal{S}} = \infty $, contradiction.
\end{example}

\begin{corollary}
If $ \mathbb{X} $ is irreducible, then with probability 1, we have
\[ \mathbb{X} \text{ is positive recurrent}
  \Leftrightarrow \mathbb{X} \text{ admits a stationary distribution.} \]

\begin{proof}[proof ($ \Rightarrow $)]
By \autoref{thm:long-run}, $ r_{i} > 0 $ exists for each $ i \in \mathcal{S} $. In addition, by \autoref{lem:r-eq-rp}, $ \mathbf{r}\mathbb{P} = \mathbf{r} $. Therefore,
\[ \frac{r}{\sum_{i \in \mathcal{S}} r_{i}} \]
is a stationary distribution of $ \mathbb{X} $.
\end{proof}

\begin{proof}[proof ($ \Leftarrow $)]
If $ \mathbb{X} $ is not positive recurrent, then $ r_{i} = 0 $ for all $ i \in \mathcal{S} $ with probability 1.

Since $ \mathbb{X} $ admits stationary distribution, we know that $ \boldsymbol\pi = \mathbf{r} $ (by \autoref{lem:pi-eq-r}), and hence $ \pi_{i} = 0 $ for all $ i \in \mathcal{S} $. That is, $ \sum_{i \in \mathcal{S}} \pi_{i} = 0 $, contradiction.
\end{proof}
\end{corollary}
