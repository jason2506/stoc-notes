\section{Recurrent and Transient}

\begin{definition}
$ f_{i} = P(\exists t \ge 1. X(t) = i \mid X(0) = i) $ is called \defterm{recurrent probability} of state $ i $ {\small \color{gray} (時間 $ 0 $ 從 $ i $ 出發,最終會回到 $ i $ 的機率)}.
\begin{itemize}
  \item state $ i $ is \defterm{recurrent} if $ f_{i} = 1 $,
  \item state $ i $ is \defterm{transient} if $ f_{i} < 1 $,
  \item $ \mathbb{X} $ is \defterm*{recurrent} if all states of $ \mathbb{X} $ are recurrent,
  \item $ \mathbb{X} $ is \defterm*{transient} if all states of $ \mathbb{X} $ are transient.
\end{itemize}
\end{definition}

\begin{observation}
Let $ T_{i} $ be the number of time indices $ t $ with $ X(t) = i $, including time $ 0 $ (support that $ X(0) = i $),
\[ E[T_{i} \mid X(0) = i] = \sum_{t = 0}^{\infty} \mathbb{P}^{t}[i, i]. \]

\begin{proof}
Let
\[
T^{(t)}_{i} =
  \begin{cases}
    1 & \text{if } X(t) = i, \\
    0 & \text{otherwise.}
  \end{cases}
\]
we get
\[ T_{i} = \sum_{t = 0}^{\infty} T^{(t)}_{i}. \]
Therefore,
\begin{eqnarray*}
E[T_{i} \mid X(0) = i]
  & = & \sum_{t = 0}^{\infty} E[T^{(t)}_{i} \mid X(0) = i] \\
  & = & \sum_{t = 0}^{\infty} P[T^{(t)}_{i} = 1 \mid X(0) = i] \\
  & = & \sum_{t = 0}^{\infty} P[X(t) = i \mid X(0) = i] \\
  & = & \sum_{t = 0}^{\infty} \mathbb{P}^{t}[i, i].
\end{eqnarray*}
\end{proof}
\end{observation}

\begin{lemma}
State $ i $ is recurrent iff $ \sum_{t = 0}^{\infty} \mathbb{P}^{t}[i, i] = \infty $.
\begin{proof}[proof $ (\Rightarrow) $]
\begin{eqnarray*}
\sum_{t = 0}^{\infty} \mathbb{P}^{t}[i, i]
  & = & E[T_{i} \mid X(0) = i] \\
  & = & \sum_{t = 0}^{\infty} k \cdot P(T_{i} = k \mid X(0) = i) \\
  & = & \infty.
\end{eqnarray*}
\end{proof}

\begin{proof}[proof $ (\Leftarrow) $]
Assume that $ f_{i} < 1 $ and hence $ P(T_{i} = k \mid X(0) = i) = f_{i}^{k - 1} \cdot (1 - f_{i}) $. We know that $ T_{i} $ is a geometic distribution with success probability $ (1 - f_{i}) $. Thus,
\[ E[T_{i} | X(0) = i] = \frac{1}{1 - f_{i}} < \infty, \]
contradiction.
\end{proof}
\end{lemma}

\begin{corollary} \label{cor:finite-X-has-recurrent-state}
Every finite $ \mathbb{X} $ has at least one recurrent state.
\begin{proof}
If there is no recurrent state, then $ E[T_{i} \mid X(0) = i] $ is finite for all state $ i $, and hence the excepted number of steps is finite. Contradiction.
\end{proof}
\end{corollary}

\begin{definition}
States $ i, j $ \defterm{communicate} if there are $ m $ and $ n $ s.t.
\[ \mathbb{P}^{m}[j, i] \cdot \mathbb{P}^{n}[i, j] > 0, \]
written as $ i \leftrightarrow j $.
\end{definition}

\begin{lemma} \label{lem:communicate}
If $ i \leftrightarrow j $, then
\begin{itemize}
  \item $ i $ is recurrent iff $ j $ is recurrent,
  \item $ i $ is transient iff $ j $ is transient.
\end{itemize}
\begin{proof}[proof ($ i $ is recurrent $ \Rightarrow $ $ j $ is recurrent)]
\begin{eqnarray*}
\sum_{t = 0}^{\infty} \mathbb{P}^{t}[j, j]
  & \ge & \sum_{t = 0}^{\infty} \mathbb{P}^{t + m + n}[j, j] \\
  & \ge & \sum_{t = 0}^{\infty} \mathbb{P}^{m}[j, i] \cdot \mathbb{P}^{t}[i, i] \cdot \mathbb{P}^{n}[i, j] \\
  & = & \mathbb{P}^{m}[j, i] \cdot \mathbb{P}^{t}[i, i] \cdot \underbrace{\sum_{t = 0}^{\infty} \mathbb{P}^{n}[i, j]}_{= \infty} \\
  & = & \infty.
\end{eqnarray*}
\end{proof}
\end{lemma}

\begin{definition}
$ \mathbb{X} $ is \defterm{irreducible} if $ i \leftrightarrow j $ for all states $ i, j $ of $ \mathbb{X} $.
\end{definition}

\begin{corollary}
If $ \mathbb{X} $ is finite and irreducible, then $ \mathbb{X} $ is recurrent.
\begin{proof}
Immedate from \autoref{cor:finite-X-has-recurrent-state} and \autoref{lem:communicate}.
\end{proof}
\end{corollary}

\begin{example}
All of states are recurrent.
\begin{multicols}{2}
\[ \bordermatrix{
    & 1 & 2 & 3 & 4 \cr
  1 & 0 & 0 & + & + \cr
  2 & + & 0 & 0 & 0 \cr
  3 & 0 & + & 0 & 0 \cr
  4 & 0 & + & 0 & 0
} \]

\null \vfill
\null \vfill
\begin{tikzpicture}
  \node[state] (A)              {$ 1 $};
  \node[state] (B) [right of=A] {$ 2 $};
  \node[state] (C) [right of=B] {$ 3 $};
  \node[state] (D) [right of=C] {$ 4 $};

  \path (A) edge [bend left=40,] (C)
            edge [bend right=50] (D)
        (B) edge [bend left=40] (A)
        (C) edge [bend left=40] (B)
        (D) edge [bend left=45] (B);
\end{tikzpicture}
\end{multicols}
\end{example}

\begin{example}
State 5 is transient, all other states are recurrent.
\begin{multicols}{2}
\[ \bordermatrix{
    & 1 & 2 & 3 & 4 & 5 \cr
  1 & + & + & 0 & 0 & 0 \cr
  2 & + & + & 0 & 0 & 0 \cr
  3 & 0 & 0 & + & + & 0 \cr
  4 & 0 & 0 & + & + & 0 \cr
  5 & + & + & 0 & 0 & +
} \]

\null \vfill
\null \vfill
\begin{tikzpicture}
  \node[state] (A)              {$ 1 $};
  \node[state] (B) [right of=A] {$ 2 $};
  \node[state] (C) [below of=A] {$ 3 $};
  \node[state] (D) [right of=C] {$ 4 $};
  \node[state] (E) [right of=B] {$ 5 $};

  \path (A) edge [loop left] (A)
            edge [bend left=40] (B)
        (B) edge [bend left=40] (A)
            edge [loop right] (B)
        (C) edge [loop left] (C)
            edge [bend left=40] (D)
        (D) edge [bend left=40] (C)
            edge [loop right] (D)
        (E) edge [bend right=50] (A)
            edge [bend right=40] (B);
\end{tikzpicture}
\end{multicols}
\end{example}
