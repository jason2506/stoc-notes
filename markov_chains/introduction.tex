\section{Introduction}

\begin{definition}
A \keyterm{stochastic process} $ \mathbb{X} $ with \keyterm{state space} $ \mathcal{S} $ is a \defterm{Markov chain} if there exist numbers $ p_{ij} $ for any states $ i, j \in \mathcal{S} $ such that 
\begin{itemize}
\item $ 0 \le p_{ij} \le 1 $,
\item $ \sum_{j \in \mathcal{S}} p_{ij} = 1 $,
\item $ p_{ij} = P(X(t + 1) = j \mid X(0) = i_{0}, X(1) = i_{1}, \cdots, X(t) = i_{i}). $
\end{itemize}

\begin{comment}
The above conditional probability is independent $ i_{0}, \cdots, i_{t - 1} $, and $ t $. Therefore,
\[ P(X(t + 1) = j \mid X(t) = i) = P(X(t + l + 1) = j \mid X(t + l) = i_{t}). \]
\end{comment}
\end{definition}

\begin{definition}
$ \mathbb{P} = [ p_{ij} ] $, $ i, j \in \mathcal{S} $ is called \defterm{matrix of transition probabilities}.
$ \mathbb{P}^{(n)} = \mathbb{P}^{n} = [ p^{(n)}_{ij} ] $ is called \defterm{matrix of $ n $-step transition probabilities}.
\begin{eqnarray*}
p^{(n + 1)}_{ij}
  & = & P(X(n + 1) = j \mid X(0) = i) \\
  & = & \sum_{k \in \mathcal{S}} P(X(n + 1) = j \mid X(1) = k, X(0) = i) \cdot P(X(1) = k \mid X(0) = i) \\
  & = & \sum_{k \in \mathcal{S}} \mathbb{P}^{n}[k, j] \cdot \mathbb{P}[i, k] \\
  & = & \text{inner product of the $ i $-th row of $ \mathbb{P} $ and $ j $-th column of $ \mathbb{P}^{n} $} \\
  & \Rightarrow & \mathbb{P}^{n + 1} = \mathbb{P} \cdot \mathbb{P}^{n}.
\end{eqnarray*}
\begin{comment}
There are several properties about a matrix of transition probabilities:
\begin{itemize}
\item $ \mathbb{P} $ models $ \mathbb{X} $,
\item any row summation of $ \mathbb{P} = 1 $,
\item $ 0 \le \text{column summation of }\mathbb{P} \le \text{size of }\mathcal{S} $.
\end{itemize}
\end{comment}
\end{definition}

\begin{example} 
一個袋子中有兩顆球(藍綠各一顆),每次取出一球後以 $ 0.8 $ 的機率原球放回袋中;以 $ 0.2 $ 的機率換成另一個顏色的球放入袋中。

問:第五次取出綠球的機率是?又假設一開始的袋子中是兩顆藍球,則第五次取出綠球的機率是?

令 $ G(t) = $ 取第 $ t $ 次之後袋子中綠球的數量,則與 $ G $ 對應之 matrix of transition probabilities 如下
\begin{multicols}{2}
\[ \mathbb{P} = \bordermatrix{
    &  0 & 1  &  2 \cr
  0 & .8 & .2 &  0 \cr
  1 & .1 & .8 & .1 \cr
  2 &  0 & .2 & .8 
} \]
\null \vfill
\null \vfill
\begin{tikzpicture}
  \node[state] (X0)              {$ 0 $};
  \node[state] (X1) [right of=X0] {$ 1 $};
  \node[state] (X2) [right of=X1] {$ 2 $};
  \path (X0) edge [loop above] node {$0.8$} (X0)
        (X1) edge [loop above] node {$0.8$} (X1)
        (X2) edge [loop above] node {$0.8$} (X2);
  \path (X0) edge [bend left=20] node {$0.2$} (X1)
        (X1) edge [bend left=20] node {$0.1$} (X0);
  \path (X2) edge [bend left=20] node {$0.2$} (X1)
        (X1) edge [bend left=20] node {$0.1$} (X2);
\end{tikzpicture}
\end{multicols}

答:
\[A_1 = \mathbb{P}^4[1,2] \cdot 1 + \mathbb{P}^4[1,1] \cdot \frac{1}{2} + \mathbb{P}^4[1,0] \cdot 0 = 0.5\]
\[A_2 = \mathbb{P}^4[0,2] \cdot 1 + \mathbb{P}^4[0,1] \cdot \frac{1}{2} + \mathbb{P}^4[0,0] \cdot 0 = 0.2952\]
\end{example}

\begin{question} \label{q:balls}
8 個桶子 9 顆球,每次投一顆球進任何一個桶子的機率都是 $ \frac{1}{8} $。問:丟完九個球之後,只有三個桶子裡有球的機率?
\[ \mathbb{P} = \bordermatrix{
    & 1 & 2 & 3 & 4 & 5 & 6 & 7 & 8 \cr
  1 & \frac{1}{8} & \frac{7}{8} & 0 & 0 & 0 & 0 & 0 & 0 \cr
  2 & 0 & \frac{2}{8} & \frac{6}{8} & 0 & 0 & 0 & 0 & 0 \cr
  3 & 0 & 0 & \frac{3}{8} & \frac{5}{8} & 0 & 0 & 0 & 0 \cr
  4 & 0 & 0 & 0 & \frac{4}{8} & \frac{4}{8} & 0 & 0 & 0 \cr
  5 & 0 & 0 & 0 & 0 & \frac{5}{8} & \frac{3}{8} & 0 & 0 \cr
  6 & 0 & 0 & 0 & 0 & 0 & \frac{6}{8} & \frac{2}{8} & 0 \cr
  7 & 0 & 0 & 0 & 0 & 0 & 0 & \frac{7}{8} & \frac{1}{8} \cr
  8 & 0 & 0 & 0 & 0 & 0 & 0 & 0 & \frac{8}{8}
} \]
\begin{figure}[htp]
\centering
\begin{tikzpicture}
  \node[state] (X1)              {$ 1 $};
  \foreach \x in {2,...,8}{
    \pgfmathtruncatemacro{\y}{\x - 1}
    \node[state] (X\x) [right of=X\y] {$ \x $};
  }

  \path (X8) edge [loop above] (X8);
  \foreach \x in {2,...,8}{
    \pgfmathtruncatemacro{\y}{\x - 1}
    \path (X\y) edge [bend right=0] (X\x)
          (X\y) edge [loop above] (X\y);
  }
\end{tikzpicture}
\end{figure}

答:$ \mathbb{P}^{8} [1, 3] $。
\end{question}

% ===============================================
%   State Space Reduction
% ===============================================
\subsection{State Space Reduction}

Say that $ A $ is a set of \keyterm{accepting states}, we want to know
\[ P(X(t) \in A \mid X(0) = i), \]
we can make a \keyterm{abosrbing state} $ a $, and let
\[ \mathbb{Q}[i, j] = \begin{cases}
  \mathbb{P}[i, j]                & \text{if } a \not\in \{i, j\}, \\
  1                               & \text{if } i = a = j, \\
  \sum_{k \in A} \mathbb{P}[i, k] & \text{if } i \neq a, j = a, \\
  0                               & \text{otherwise.}
\end{cases} \]

\begin{comment}
An accepting states, $ A $, is built according to the question. 
\end{comment}

\begin{example}
\autoref{q:balls} 中,由於 state 4 到 8 回到前 3 個 state 的機率皆為 0,所以我們可以將 state 4 到 8 合併成 absorbing state $ a $。
\[ \mathbb{Q} = \bordermatrix{
    & 1 & 2 & 3 & a \cr
  1 & \frac{1}{8} & \frac{7}{8} & 0 & 0 \cr
  2 & 0 & \frac{2}{8} & \frac{6}{8} & 0 \cr
  3 & 0 & 0 & \frac{3}{8} & \frac{5}{8} \cr
  a & 0 & 0 & 0 & 1 \cr
} \]
\end{example}

\begin{example}
假設隨機銀行中有個不為人知的帳戶 $ X $(裡面已經有 $ 3 $ 萬元)。每個月該帳戶都會有神秘善心人士存入 $ 2 $ 萬,但是同時會被竊賊偷走 $ 1 $、$ 2 $、$ 3 $ 或 $ 4 $ 萬元(偷多少錢的機率是一樣的)。另外,每個月月底結算之時隨機政府會將帳戶中超過 $ 3 $ 萬的金額全部視為存款稅沒收。問:前四個月該帳戶中的存款曾經少於等於 $ 1 $ 萬元的機率?
\begin{description}
  \item[觀察 1] 帳戶中可能的金額為 $ \mathcal{S} = \{3, 2, 1, 0, -1, -2, \cdots \} $(萬元)。
  \item[觀察 2] 根據問題可知 accepting states $ \{1, 0, -1, -2, \cdots\} = A $。
\end{description}

\[ \mathbb{Q} = \bordermatrix{
    & a   & 2   & 3   \cr
  a & 1   & 0   & 0   \cr
  2 & \frac{2}{4}  & \frac{1}{4} & \frac{1}{4} \cr
  3 & \frac{1}{4} & \frac{1}{4} & \frac{2}{4}
} \]

答:
\[ P(\exists t \in \{0, 1, \cdots, n \} . X(t)\in A \mid X(0) = 3) = \mathbb{Q}^{t}[3, a]. \]

因此,前四個月曾經小於等於 $ 1 $ 萬 $ = \mathbb{Q}^{4}[3, a] = 0.7851 $。

\end{example}